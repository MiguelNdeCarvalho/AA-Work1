\documentclass[11pt]{article}   % tipo de documento e tamanho das letras

% os seguintes pacotes estendem a funcionalidade básica:
\usepackage[a4paper, total={16cm, 24cm}]{geometry} % tamanho da pagina e do texto
\usepackage[portuguese]{babel}  % traduz para portugues
\usepackage[utf8]{inputenc}
\usepackage{graphicx}           % graficos
\usepackage{amsmath}            % matematica
\usepackage{tikz}               % diagramas
    \usetikzlibrary{shadows}
\usepackage{booktabs}           % tabelas com  melhor aspecto
\usepackage[colorlinks=true]{hyperref}           % links para partes do documento ou para a web
\usepackage{listings}           % incluir codigo
    \renewcommand\lstlistingname{Listagem}  % Listing em portugues
    \lstset{numbers=left, numberstyle=\tiny, numbersep=5pt, basicstyle=\footnotesize\ttfamily, frame=tb,rulesepcolor=\color{gray}, breaklines=true}
\usepackage{blindtext}

% -------------------------------------------------------------------------------------------
\title
{
    \includegraphics[width=0.3\textwidth]{images/logo_universidade.png}
    \\[0.1cm]
    \textbf{Gerador de Árvores} \\
    Aprendizagem Automática
}

\author
{
    \textbf{Professor:} Luís Rato \\
    \textbf{Realizado por:} Miguel de Carvalho (43108) \\ João Pereira (42864) 
}
\date{\today}

% -------------------------------------------------------------------------------------------
%                                Body                                                       %
% -------------------------------------------------------------------------------------------

\begin{document}
\maketitle

% -------------------------------------------------------------------------------------------
\section{Introdução} 

\hspace{0,5cm}Neste trabalho foi solicitado a realização de um programa que simule o \textbf{Escalonamento de Processos} num Modelo de 3 Estados, representado na figura abaixo. \par
\begin{figure}[h!]
    \begin{center}
        \includegraphics[width=0.5\textwidth]{images/states.png}
        \caption{Modelo de 3 Estados}
    \end{center}
\end{figure}
O \textbf{Escalonador de Processos} faz parte do \textbf{Sistema Opertivo} e é responsável por decidir em que momento cada processo estará no CPU. 
Existem muitos algoritmos de escalonamento para realizar essa decisão. \par
Neste trabalho serão utilizados o algoritmo \textbf{FCFS} e o \textbf{Round Robin (RR)}:
\begin{itemize}
    \item O \textbf{FCFS} é um algoritmo de escalonamento não preemptivo que prioriza os processos pela ordem de chegada. Executa o processo todo do inicio ao fim sem o interromper, até estar concluído. Quando aparece um novo processo e ainda existe um em execução, esse novo fica em fila de espera.
    \item O \textbf{Round Robin (RR)} é um algoritmo de escalonamento preemptivo que apresenta um funcionamento igual ao do \textbf{FCFS}, mas com tempo limite de execução, o \textbf{Quantum}. Assim, quando o processo se encontra em execução este será interrompido quando o tempo de execução for igual ao \textbf{Quantum} e fica em fila de espera (\textbf{READY}). 
\end{itemize}

% -------------------------------------------------------------------------------------------
% -------------------------------------------------------------------------------------------
\newpage
\section{Conclusão} % Conclusão
\hspace{0,5cm}Em suma, com a realização deste trabalho "Simulador de Escalonamento" fiquei muito mais esclarecido sobre o seu funcionamento. \par
Saliento que me ajudou a entender como funciona o escalonador e as condições que cada algoritmo (\textbf{FCFS}/\textbf{RR}), usa para proceder à mudança dos processos entre os estados e as respetivas diferenças de tempo no mesmo conjunto de processos, entre os respetivos algoritmos.  
% -------------------------------------------------------------------------------------------
\end{document}
